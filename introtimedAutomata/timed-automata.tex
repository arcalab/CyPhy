% !TeX program = xelatex
\documentclass{beamer}
\usepackage{etex} % fixes new-dimension error
%-------------- template --------------------------------------------------
\usetheme{metropolis}
\metroset{block=fill}
%\usetheme{Boadilla}

% Configuring the foot line
\setbeamertemplate{footline}
{
  \leavevmode%
  \hbox{%
  \begin{beamercolorbox}[wd=.4\paperwidth,ht=2.25ex,dp=1ex,center]{author in head/foot}%
    \usebeamerfont{author in head/foot}\insertshortauthor
  \end{beamercolorbox}%
  \begin{beamercolorbox}[wd=.5\paperwidth,ht=2.25ex,dp=1ex,center]{title in head/foot}%
    \usebeamerfont{title in head/foot}\insertsection
  \end{beamercolorbox}%
  \begin{beamercolorbox}[wd=.1\paperwidth,ht=2.25ex,dp=1ex,right]{date in head/foot}%
    \insertframenumber{} / \inserttotalframenumber\hspace*{2ex} 
  \end{beamercolorbox}}%
  \vskip0pt%
}
% No configuration symbols
\makeatother
\setbeamertemplate{navigation symbols}{}
%----------------------------------------------------------------------------
% context
\AtBeginSection[]
{
    \begin{frame}
        \frametitle{Table of Contents}
        \tableofcontents[currentsection]
    \end{frame}
}
\author[Renato Neves]{Renato Neves}

% logos of institutions
\titlegraphic{
  \begin{textblock*}{5cm}(7.8cm,7.45cm)
     \includegraphics[scale=0.044]{images/uminho.png}\hspace*{.85cm}~%
  \end{textblock*}
  \begin{textblock*}{5cm}(9.8cm,7.45cm)
    \includegraphics[scale=0.4]{images/haslab.pdf}
  \end{textblock*}
}

% no date
\date{}


%----------------------------------------------------------------------------
\usepackage{graphicx,amsmath}
\usepackage{stmaryrd} % cf. interleave
\usepackage{booktabs}
\usepackage{amscd}
\usepackage[absolute,overlay]{textpos} % this is for textblock
\usepackage{tikz}
\usetikzlibrary{arrows.meta, calc, fit, tikzmark}
\usepackage{pgfplots}
\usepackage{alltt}
\usepackage{listings}

\def\pvo#1#2{\langle \! \! \! \langle #1 \rangle \! \! \! \rangle\, #2}
\def\uppaal{\textsc{Uppaal}}
\def\cc#1{\mathcal{C}(#1)}
\def\TL#1{\mathcal{T}(#1)}
\input{macros/macros}



\begin{document}

\title{Timed Automata}

\frame[plain]{\titlepage}

\begin{frame}{Last lecture}
  Visited \alert{syntax} and \alert{semantics}

  Analysed central ideas in \alert{concurrency} and
  \alert{synchronisation}

  \pause
  \bigskip
  We will now see how \alert{time} fits in
\end{frame}
\section{Motivation}
%----------------------------------------------------------------------------------
\begin{slide}{Motivation}


Saying that an airbag in a car crash \alert{eventually
inflates} is insufficient -- it would be better to say that \dots
\begin{center}
\dots\ in a car crash the airbag inflates \alert{within 20ms}
\end{center}
 
\vfill
\emph{Correctness in time-critical systems not only depends on
      the logical result of the computation, but also \alert{on the time
      at which the results are produced}}
\begin{flushright}
[Baier \& Katoen, 2008]
\end{flushright}
\end{slide}

\begin{slide}{Examples of time-critical systems}

\begin{block}{Traffic lights}
  Lights activate at specific time intervals
\end{block}

\medskip
\begin{block}{(Re)transmission  protocols}
  Communication of large files between a remote unit and a video/audio
  equipment 
\end{block}

\medskip
\begin{block}{Many others}
Pacemakers, autonomous driving, electric grids
\dots
\end{block}
\end{slide}

\begin{slide}{This chapter}
        We will explore an \alert{automaton-based formalism} with an explicit
        notion of a \alert{clock} 
\begin{center}
\fbox{Timed Automata [Alur \& Dill, 90]}
\end{center}


\bigskip
Associated tool
\begin{itemize}
\item \uppaal\ [Behrmann, David, Larsen, 04]
\end{itemize}
\end{slide}

\begin{slide}{\uppaal}

\begin{center}
\fbox{\uppaal\ = (\alert{Upp}sala University + \alert{Aal}borg University) [1995]}
\end{center}

\begin{itemize}
\item Toolbox for modelling and analysis of timed systems
\item Systems modelled as \alert{networks} of timed automata 
        with \alert{channel synchronisations}
\item Properties specified in a \alert{temporal logic} 
\end{itemize}

\end{slide}

\section{The very basics of timed automata}


\begin{slide}{Timed automata}
Finite-state machines equipped with \alert{real-valued clocks}

\begin{itemize}
\item clocks can only be read or
\item reset to zero (after which they start increasing their value again as time progresses)
\item a clock's value corresponds to time elapsed since its last reset 
\item all clocks proceed synchronously (\emph{i.e.} at the same rate)
\end{itemize}
\end{slide}
% % ----------------------------------------------------------------------------------

\begin{frame}{Example: the annoying lamp}
  
\begin{figure}[htb]
  \centering
  \includegraphics[scale=0.35]{./images/Lamp.pdf}\\
\end{figure}

\centering
{\scriptsize (extracted from \uppaal)}

\end{frame}

\begin{slide}{Guards, updates, and invariants}
\small \centering

\begin{tabular}{c}
   \includegraphics[width=8cm]{./images/model0.jpg} \\  \includegraphics[width=7cm]{./images/model1.jpg}
\end{tabular}

\end{slide}

\begin{slide}{Timed automata}
\small
A timed automaton is a tuple
$\pair{\mathbf{L}, \mathbf{L_0}, \mathbf{Act}, \mathbf{C}, \mathbf{Tr}, 
\mathbf{Inv}}$ 
\begin{itemize}
     \item $\mathbf{L}$ a set of locations and 
             $\mathbf{L_0} \subseteq \mathbf{L}$ set
        of \alert{initial} locations
\item $\mathbf{Act}$ set of actions (\alert{channels}) 
        and $\mathbf{C}$ set of clocks
     \item
             $\mathbf{Tr} \subseteq \mathbf{L} 
             \times \cc{\mathbf{C}} \times \mathbf{Act} 
             \times \pow{(\mathbf{C})} \times \mathbf{L}$
     is a \alert{transition} relation
\begin{equation*}
  \ell_1\; \tran{g,a,U}\;  \ell_2
\end{equation*}
transition from location $\ell_1$ to $\ell_2$, labelled by $a$, enabled if
\alert{guard} $g$ holds; when performed resets the set $U$ of clocks
\item $\fdec{\mathbf{Inv}}{\mathbf{L}}{\cc{\mathbf{C}}}$ assignment of invariants to
  locations
\end{itemize}
$\cc{\mathbf{C}}$ denotes the set of clock constraints over a set $C$ of
clocks
\end{slide}

\begin{frame}{A revisit of the annoying lamp}
  
\begin{figure}[htb]
  \centering
  \includegraphics[scale=0.35]{./images/Lamp.pdf}\\
\end{figure}

Exercise: define $\pair{L, L_0, Act, C, Tr, Inv}$ 

\end{frame}

\section{Parallel Composition}

\begin{frame}{Parallel composition of timed automata}

        Communication mechanism analogous to CCS

        \dots\ but requires extra machinery
\end{frame}

\begin{slide}{Definition}

  Let $H = Act_1 \cap Act_2$. The parallel composition of
  $ta_1$ and $ta_2$ synchronising on $H$ is the timed automaton

  \vspace{0.2cm}
$ta_1 \parallel_H ta_2 := \pair{L_1 \times L_2, L_{0,1} \times L_{0,2}, 
\mathbf{Act}, C_1 \cup C_2, \mathbf{Tr}, 
\mathbf{Inv}}$

  \vspace{0.2cm}
\begin{itemize}
\item $\mathbf{Act} = ((Act_1 \cup Act_2) - H) \cup \enset{\tau}$
\item $\mathbf{Inv} (\ell_1,\ell_2) = Inv_1(\ell_1) \e  Inv_2(\ell_2)$
\item $\mathbf{Tr}$ is given by:
\begin{itemize}
\item $(\ell_1,\ell_2) \tran{g,a,U} (\ell'_1,\ell_2)\; $ if $\; a \not \in H \e  \ell_1 \tran{g,a,U} \ell'_1 $
\item $(\ell_1,\ell_2) \tran{g,a,U} (\ell_1,\ell'_2)\; $ if $\; a \not \in H \e   \ell_2 \tran{g,a,U} \ell'_2$
\item $(\ell_1,\ell_2) \tran{g,\tau,U} (\ell'_1,\ell'_2)\; $ if $\; a \in H \e  \ell_1 \tran{g_1,a,U_1} \ell'_1 \e \ell_2 \tran{g_2,\overline{a},U_2} \ell'_2$\\
with $g = g_1 \e g_2$ and $U = U_1 \cup U_2$
\end{itemize}
\end{itemize}
\end{slide}

\begin{slide}{Example: (re)revisiting the lamp interrupt}

\begin{figure}[htb]
  \centering
  \includegraphics[scale=0.25]{./images/Lamp.pdf}
  \includegraphics[scale=0.09]{./images/User.pdf}\\
\end{figure}

\end{slide}

% %----------------------------------------------------------------------------------
\begin{slide}{Exercise: worker, hammer, nail}
        Write down the parallel composition of the following automata

\begin{figure}[htb]
  % Requires \usepackage{graphicx}
  \includegraphics[width=50mm]{./images/Worker.pdf}
  \hspace{0.5cm}
  \includegraphics[width=50mm]{./images/Hammer.pdf}

  \vspace{0.5cm}
  \includegraphics[width=50mm]{./images/Nail.pdf}
\end{figure}
\end{slide}
\end{document}
